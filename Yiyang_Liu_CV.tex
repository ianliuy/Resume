%-------------------------
% Resume in Latex
% Author :  yiyangiliu (https://github.com/yiyangiliu)
% Thanks to: Jake Gutierrez (https://github.com/jakegut/resume)
%            Sourabh Bajaj (https://github.com/sb2nov/resume)
% License : MIT
%------------------------

\documentclass[letterpaper,11pt]{article}
\usepackage[T1]{fontenc}
\fontfamily{Georgia}\selectfont
\fontseries{m}\selectfont
% \fontseries{bx}\selectfont
\usepackage{latexsym}
\usepackage{microtype}
\usepackage[empty]{fullpage}
\usepackage{titlesec}
\usepackage{marvosym}
\usepackage[usenames,dvipsnames]{color}
\usepackage{verbatim}
\usepackage{enumitem}
\usepackage[hidelinks]{hyperref}
\usepackage{fancyhdr}
\usepackage[english]{babel}
\usepackage{tabularx}
\input{glyphtounicode}

\renewcommand{\bfseries}{\fontseries{b}\selectfont}

%----------FONT OPTIONS----------
% sans-serif
% \usepackage[sfdefault]{FiraSans}
% \usepackage[sfdefault]{roboto}
% \usepackage[sfdefault]{noto-sans}
% \usepackage[default]{sourcesanspro}

% serif
% \usepackage{CormorantGaramond}
% \usepackage{charter}


\pagestyle{fancy}
\fancyhf{} % clear all header and footer fields
\fancyfoot{}
\renewcommand{\headrulewidth}{0pt}
\renewcommand{\footrulewidth}{0pt}

% Adjust margins
\addtolength{\oddsidemargin}{-0.5in}
\addtolength{\evensidemargin}{-0.5in}
\addtolength{\textwidth}{1in}
\addtolength{\topmargin}{-.55in}
\addtolength{\textheight}{1.0in}

\urlstyle{same}

\raggedbottom
\raggedright
\setlength{\tabcolsep}{0in}

% Sections formatting
\titleformat{\section}{
    \vspace{-4pt}\raggedright\small\bfseries
}{}{0em}{}[\color{black}\titlerule \vspace{-5pt}]

% Ensure that generate pdf is machine readable/ATS parsable
\pdfgentounicode=1

%-------------------------
% Custom commands

\newcommand{\resumeItem}[1]{
    \item\small{
            {#1 \vspace{-2pt}}
    }
}

\newcommand{\resumeSubheading}[4]{
    \vspace{-2pt}\item
    \begin{tabular*}{\textwidth}[t]{l@{\extracolsep{\fill}}r}
        \normalsize\textbf{#1} & #2 \\
        \textit{\small#3} & \textit{\small #4} \\
    \end{tabular*}\vspace{-7pt}
}

\newcommand{\resumeSubSubheading}[2]{
    \item
    \begin{tabular*}{\textwidth}{l@{\extracolsep{\fill}}r}
        \textit{\small#1} & \textit{\small #2} \\
    \end{tabular*}\vspace{-7pt}
}

\newcommand{\resumeProjectHeading}[2]{
    \item
    \begin{tabular*}{\textwidth}{l@{\extracolsep{\fill}}r}
        \small#1 & #2 \\
    \end{tabular*}\vspace{-7pt}
}

\newcommand{\resumeSubItem}[1]{\resumeItem{#1}\vspace{-4pt}}

\renewcommand\labelitemii{$\vcenter{\hbox{\scriptsize$\bullet$}}$}

\newcommand{\resumeSubHeadingListStart}{\begin{itemize}[leftmargin=0in, label={}]}
\newcommand{\resumeSubHeadingListEnd}{\end{itemize}}
\newcommand{\resumeItemListStart}{\begin{itemize}[leftmargin=.4in, labelsep=.13in]}
\newcommand{\resumeItemListEnd}{\end{itemize}\vspace{-5pt}}

%-------------------------------------------
%%%%%%  RESUME STARTS HERE  %%%%%%%%%%%%%%%%%%%%%%%%%%%%


\begin{document}

%----------HEADING----------

\begin{center}
{\LARGE \scshape \bfseries Yiyang Liu} \\ \vspace{1pt}
3650 McClintock Ave, Los Angeles, CA 90089 \\ \vspace{1pt}
\small +1 222-333-333 $|$ \href{mailto:233@usc.edu}{\underline{233@usc.edu}} $|$
\href{https://linkedin.com/in/yiyangiliu}{\underline{linkedin.com/in/yiyangiliu}} $|$
\href{https://github.com/yiyangiliu}{\underline{github.com/yiyangiliu}}
\end{center}


%-----------EDUCATION-----------
\section{EDUCATION}
\resumeSubHeadingListStart
\resumeSubheading
{University of Southern California}{Los Angeles, CA}
{MS, Applied Data Science}{Aug. 2021 -- May 2023}
\resumeSubheading
{Beijing University of Technology}{Beijing, China}
{BE, Computer Science (with Hons.); GPA: 2.33/4, WAM: 23.3/100, Ranking: 23/233}{Sep. 2016 -- July 2020}
\resumeSubHeadingListEnd


%-----------EXPERIENCE-----------
\section{PROFESSIONAL EXPERIENCE}
\resumeSubHeadingListStart

\resumeSubheading
{ByteDance}{Beijing, China}
{Software Engineer Intern}{Feb. 2021 -- July 2021}
\resumeItemListStart
\resumeItem{\textbf{Action Center:} 自动化将用户行为从数据库灌入行为中心,提高推荐系统的工作效率。从数据库读取事件,转化为用户行为数据,向行为中心发送远程过程调用请求,并记录日志。使用流式计算工人加快效率,批处理,容器加快部署。Automation will use user behavior to enter the behavior center in the database, and improve the efficiency of the recommended system.}
\resumeItem{\textbf{User Event Storage:} 使用流式计算将用户在客户端的行为行为产生的日志,通过字符串解析,转化为各种事件。将事件字符串转化为结构体后存入数据库,产生天级用户事件数据库。使用引入外部配置。使用脚本控制二进制文件,自动化建库流程Automatically parse events from the log, generate a stored database file, and deploy the service will behavior.The log generated by the user's behavior on the client using streaming computing}
\resumeItem{\textbf{A/B Test Toolkit:} 为算法工程师提供切换模型的对照工具,使得能端到端地检查不同模型打开时对推荐系统喂出流的效果。将模型的平均交付时间缩短了百分之十Provided control tool for machine learning engineers to check different models' end-to-end effect on the feed of our recommender system. Reduced 10\% of average delivery time of deep-learning models.}
\resumeItem{\textbf{DevOps:} 搜索和推荐系统服务使用的字典存放在阿里云,收费使用。改造脚本,将字典从阿里云迁移到自有的分布式文件系统。每月节省了三千美金。Search and recommendation system services are stored in Ali Cloud, charges. Renovate the script, migrate the dictionary from Ali Cloud to its own distributed file system. Saved \$3000 per month.}
\resumeItemListEnd

% -----------Multiple Positions Heading-----------
%    \resumeSubSubheading
%     {Software Engineer I}{Oct 2014 - Sep 2016}
%     \resumeItemListStart
%        \resumeItem{Apache Beam}
%          {Apache Beam is a unified model for defining both batch and streaming data-parallel processing pipelines}
%     \resumeItemListEnd
%    \resumeSubHeadingListEnd
%-------------------------------------------

\resumeSubheading
{FinTech Startup \& Tsinghua University}{Beijing, China}
{Machine Learning Engineer Intern}{Nov. 2019 -- June 2020}
\resumeItemListStart
\resumeItem{\textbf{Sentiment Analysis}: 使用向量流动,在多台显卡上训练多个深度学习模型,使用交叉验证,伪标签,模型融合等方式调优。Use Tensorflow \& Keras training multiple depth-learning models on multiple GPUs, tuning with pseudo-label, cross-verification and model fusion.}
\resumeItem{\textbf{Paper Implementation (\href{https://github.com/yiyangiliu/BERT_Paper_Implementation}{\underline{GitHub}}):} 对照双向循环网络的论文,复现了提出的深度学习模型。包含所有组件,比如多头自注意力机制,转换器,嵌入层,全连接层,并开源代码Implemented the Bert with the paper. Contains all components, such as multi-head self attention mechanism, transformer, embedding layer. Code is open sourced.}
\resumeItem{\textbf{Financial News Dataset:} Wrote web crawler to make 23.3G financial news corpus dataset with millions of records by using Requests, Beautifulsoup, Selenium. Used Pandas to clean data.}
\resumeItem{\textbf{Kaggle Model Evaluation (\href{https://github.com/yiyangiliu/blog/blob/master/contents/kaggle-house-price-forecasting-competition-advanced.md}{\underline{GitHub}}):} 复现了卡高房价预测比赛的最好解法,评估了多种机器学习方法的指标。Implemented best solution of kaggle house-prices forecasting competition and evaluated various machine learning methods' metrics.}
\resumeItem{\textbf{Bert4Keras (\href{https://github.com/bojone/bert4keras}{\underline{GitHub}}):} 是一个著名的开源框架,提供最新的基于深度学习的语言模型。并且集成了训练、微调等方法。Bert4Keras is a prestigious open-source framework, which provides the state-of-the-art deep-learning language model like Bert. And integrate training, fine-tuning and other methods.}
% \resumeItem{\textbf{Technical Evaluation:} 为了与深度学习方法对照,评估了传统的机器学习方法,比如 Evaluated some machine learning methods, like XGBoost, in contrast with deep learning.}
\resumeItemListEnd

\resumeSubheading
{Carnegie Mellon University}{Pittsburgh, PA}
{Research Assistant}{Sep. 2019 -- Dec. 2019}
\resumeItemListStart
\resumeItem{\textbf{Cryptocurrency Transaction System (\href{https://github.com/yiyangiliu/A-simplified-Bitcoin-Transaction-System}{\underline{GitHub}}):} Mocked Bitcoin System to implenente all crucial components such as block mining, PoW (proof of work), wallet, key pairs, money transaction. Code is open sourced.}
\resumeItem{\textbf{Technical Report:} 调研了主流的比特币交易系统,写出了运行的全部流程,包括区块的挖掘,产生,证明,传播,交易等Investigated Bitcoin system, and wrote out the whole operation process, including block mining, generation, proof, dissemination, transacting, etc.}
\resumeItemListEnd

\resumeSubheading
{Ritatsu Software Tokyo}{Tokyo, Japan}
{Software Engineer Intern}{Aug. 2019 -- Sep. 2019}
\resumeItemListStart
\resumeItem{\textbf{Quality Assurance:} 设计和编写单元测试、模块测试、系统测试、端到端测试。保证向金融机构交付代码的健壮性。Designed and wrote unittests, function tests, end-to-end tests to guarantee the robustness of code delivered to some prestigious financial institutions like Nomura Securities.}
\resumeItemListEnd

\resumeSubHeadingListEnd


%-----------PROJECTS-----------
\section{SELECTED PROJECTS}
\resumeSubHeadingListStart
\resumeProjectHeading
{\textbf{RescueTime-Visualization (\href{https://github.com/yiyangiliu/RescueTime-Record}{\underline{Demo}}, \href{https://github.com/yiyangiliu/RescueTime-Visualization}{\underline{GitHub}}):}  使用接口将数据可视化,并上传到仓库,提高了个人的网络可见性Visualize RescueTime data and update it to GitHub to increase visibility.}
\resumeItemListStart
% \resumeItem{RescueTime is an automatic Time Tracker which provides in detail app usage report.}
% \resumeItem{Used RescueTime API to Generate Tables with Usage Ranking}
% \resumeItem{Automatically sent Daily Usage to Github Repository for more personal Visibility}
\resumeItemListEnd
\resumeProjectHeading
{\textbf{Deep Learning Based Advertising System (\href{https://github.com/yiyangiliu/KeystoneProject}{\underline{GitHub}}):} 使用深度学习模型识别视频中出现的物体,提高广告准确性。Use DL model to identify objects in video to increase AD accuracy.}
\resumeItemListStart
% \resumeItem{RescueTime is an automatic Time Tracker which provides in detail app usage report.}
% \resumeItem{Used RescueTime API to Generate Tables with Usage Ranking}
% \resumeItem{Automatically sent Daily Usage to Github Repository for more personal Visibility}
\resumeItemListEnd
\resumeSubHeadingListEnd
\vspace{5pt}

%
%-----------PROGRAMMING SKILLS-----------
\section{SELECTED SKILLS}
\resumeSubHeadingListStart
\resumeProjectHeading
{\textbf{Languages:} Golang, Python, Cpp, Shell}
\resumeItemListStart
% \resumeItem{RescueTime is an automatic Time Tracker which provides in detail app usage report.}
% \resumeItem{Used RescueTime API to Generate Tables with Usage Ranking}
% \resumeItem{Automatically sent Daily Usage to Github Repository for more personal Visibility}
\resumeItemListEnd
\resumeProjectHeading
{\textbf{Developer Tools:} Git, Docker, Cloud Platform, Jetbrains Suite}
\resumeItemListStart
% \resumeItem{RescueTime is an automatic Time Tracker which provides in detail app usage report.}
% \resumeItem{Used RescueTime API to Generate Tables with Usage Ranking}
% \resumeItem{Automatically sent Daily Usage to Github Repository for more personal Visibility}
\resumeItemListEnd
\resumeSubHeadingListEnd
\vspace{5pt}


%  \begin{itemize}[leftmargin=0in, label={}]
%     \small{\item{
%      \textbf{Languages}{ Golang, Python, Cpp, Shell} \\
%      \textbf{Developer Tools}{: Git, Docker, Cloud Platform, Jetbrains Suite}
%     }}
%  \end{itemize}


%-------------------------------------------
\end{document}
